\section{Master} 
\begin{itemize}
    \item Si occupa di leggere da file attraverso la funzione \enquote{scan\_data} i valori di configurazione necessari per la simulazione che verranno associati alla struct \enquote{config}. 
    \item Crea N\_ATOMI\_INIT atomi usando la funzione fork. Il processo figlio generato avrà associato un array contenente i parametri di configurazione personalizzati e il PID del processo master, eseguirà un \texttt{execvp} a \texttt{./bin/atom}.
    \item Crea il processo \texttt{psu} usando la funzione fork. Il processo figlio generato avrà associato un array contenente i parametri di configurazione personalizzati, il PID del processo master e gli ID di alcuni IPC, eseguirà un \texttt{execvp} a \texttt{./bin/psu}. 
    \item Crea il processo \texttt{inhibitor} sotto richiesta dell'utente attraverso la funzione fork. Il processo figlio generato avrà associato un array contenente i parametri di configurazione personalizzati e il PID del processo master, eseguirà un \texttt{execvp} a \texttt{./bin/inhibitor}.
    \item Si occupa di registrare i segnali che verranno gestiti dagli appositi handler, che permettono la gestione delle terminazioni della simulazione. 
    \item Dopo aver inizializzato tutto il necessario, imposta un alarm con valore \texttt{SIM\_DURATION} che rappresenta la durata complessiva della simulazione. Il segnale \texttt{SIGALARM} è gestito attraverso un handler.
    \item Generare e inviare ai processi atomo il numero atomico di riferimento 
    \item Ogni secondo mostra a schermo: 
    \begin{itemize}
        \item Il numero di fissioni avvenute nell'ultimo secondo. 
        \item Il numero di scorie raccolte nell'ultimo secondo.
        \item L'energia prodotta dei processi atomo nell'ultimo secondo.
        \item I bilanciamenti effettuati di processo inhibitor per limitare le fissioni dei processi atomo nell'ultimo secondo. 
    \end{itemize}
    \item Si occupa in seguito allo scadere di un timer visibile a schermo di far partire i processi atomi per la simulazione. 
    \item Si occupa di gestire le terminazioni della simulazione.
\end{itemize}
\section{Terminazione} 
Le terminazioni della simulazione sono gestite dal processo master attraverso la funzione \enquote{why\_term} che si occupa di interrompere la simulazione nel caso si verifichi uno dei seguenti casi  ripartiti nella richiesta progettuale, ogni terminazione è identificata da una macro che sono contenute in un enum apposito. 
Ogni volta che si presenta una terminazione avviene:
\begin{itemize}
    \item Rimozione degli oggetti IPC utilizzati dai vari processi durante la simulazione. 
    \item Rimuovere attraverso il segnale \texttt{SIGKILL} tutti i processi ancora attivi per la simulazione. 
    \item Stampare su schermo la motivazione della terminazione. 
\end{itemize}
\section{Atom}

Il processo atomo viene generato attraverso fork dal processo master ,dopo la sua nascita fa le seguenti operazioni.
    \begin{itemize}
        \item Inizializza gli \texttt{IPC} necessari.
        \item Registra il segnale \texttt{SIGCHLD} associandolo all'handler apposito. 
        \item Registra il segnale \texttt{SIGUSR1} associandolo ad un handler che si occupera in caso si verificasse \texttt{MELTDOWN} di inviare il segnale \texttt{SIGUSR1} al processo master che gestira la terminazione della simulazione. 
        \item Preleva e associa all'apposita struct i valori della configurazione attraverso argv. 
    \end{itemize}
Dopo queste operazioni il processo atomo si autoinvia un segnale \texttt{SIGSTOP} per far capire che lui ha inizializzato tutto il necessario ,attendendo dal processo atomo il segnale \texttt{SIGCONT} che arriverà nel momento in cui partirà quando scadrà il timer a schermo e verrà impostato l'alarm con \texttt{SIM\_DURATION}.
Successivamente il processo atomo si mette all'opera e dopo aver ricevuto e prelevato il comando di fissione dalla message queue che lo mette in comunicazione con il processo attivatore ,la fissione dell'atomo è gestita dalla funzione \enquote{atom\_fission} l'atomo saprà se effettuare o meno la fissione dopo avere prelevato il valore all'interno della message queue ,in caso di comando positivo : 
\begin{itemize}
    \item Nel caso in cui il suo numero atomico sarà inferiore a quello prestabilità all'interno della configurazione attraverso il parametro \texttt{MIN\_A\_ATOMICO} esso non potrà effettuare fissione e aumenterà il numero delle scorie e effettua una exit con parametro \texttt{EXIT\_SUCCESS}.
    \item Caso contrario e se il comando prelevato in message queue è uguale a 1 ,l'atomo può effettuare fissione , aumentando il numero di fissioni e generando un processo figlio attraverso funzione fork , se essa adrà a buon fine il processo figlio si occuperà di aumentare il numero di attivazioni e di INSERIRE PARTE RELATIVA AL NUMERO ATOMICO 
\end{itemize}

