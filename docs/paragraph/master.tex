\begin{itemize}
    \item Si occupa di leggere da file attraverso la funzione \lstinline{scan_data()} i valori di configurazione necessari per la simulazione che verranno associati alla struct \lstinline{config}. 
    \item Crea \textit{N_ATOMI_INIT} atomi usando la funzione fork. Il processo figlio generato avrà associato un array contenente i parametri di configurazione personalizzati e il PID del processo master, eseguirà un \textit{execvp} a \textit{./bin/atom}.
    \item Crea il processo \textit{psu} usando la funzione fork. Il processo figlio generato avrà associato un array contenente i parametri di configurazione personalizzati, il PID del processo master e gli ID di alcuni IPC, eseguirà un \textit{execvp} a \textit{./bin/psu}. 
    \item Crea il processo \textit{inhibitor} sotto richiesta dell'utente attraverso la funzione fork. Il processo figlio generato avrà associato un array contenente i parametri di configurazione personalizzati e il PID del processo master, eseguirà un \textit{execvp} a \textit{./bin/inhibitor}.
    \item Si occupa di registrare i segnali che verranno gestiti dagli appositi handler, che permettono la gestione delle terminazioni della simulazione. 
    \item Dopo aver inizializzato tutto il necessario, imposta un alarm con valore \textit{SIM_DURATION} che rappresenta la durata complessiva della simulazione. Il segnale \textit{SIGALARM} è gestito attraverso un handler.
    \item Generare e inviare ai processi atomo il numero atomico di riferimento. 
    \item Ogni secondo mostra a schermo: 
    \begin{itemize}
        \item Il numero di fissioni avvenute nell'ultimo secondo. 
        \item Il numero di scorie raccolte nell'ultimo secondo.
        \item L'energia prodotta dei processi atomo nell'ultimo secondo.
        \item I bilanciamenti effettuati di processo inhibitor per limitare le fissioni dei processi atomo nell'ultimo secondo. 
    \end{itemize}
    \item Si occupa, in seguito allo scadere di un timer visibile a schermo, di far partire i processi atomi per la simulazione. 
    \item Si occupa di gestire le terminazioni della simulazione.
\end{itemize}
