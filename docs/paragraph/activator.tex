Processo attivatore viene generanto dal processo master attraverso la chimaata alla funzione fork che ,genererà un processo figlio a cui verrà assegnato un array contentente i parametri letti da file necessari alla simulazione ,successivamente chiamerà la funzione execvp per effettuare un cambio d'immagine per diventare effettivamente un processo attivatore .
Dopo la sua nascia il proecsso attivatore effettuerà le seguenti operazioni: 
\begin{itemize}
    \item Inizializza o preleva la message queue necessaria per comunicare le fissioni al processo atomo.
    \item Genera in modo randomico un numero pari a 0 o 1 che rispettivamente hanno il seguente significato: 
        0 : Il processo atomo non effettua fissione. 
        1 : Il processo atomo ha il comando per effettuare una fissione. 
     I valori vengono generato in modo randomico e inseriti in una apposita message queue su cui gli atomi senza conoscerne precedentemente il significato preleveranno il primo messaggio disponibile che ne determinerà l'operazione di fissione di quell'atomo.
     
\end{itemize}