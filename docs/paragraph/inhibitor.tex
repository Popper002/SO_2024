Il processo inibitore viene attivato impostando sul file di configurazione il valore $1$
 a fianco di \textit{INHIBITOR}, in questo modo il processo master
genererà attraverso la funzione fork un processo figlio con un array contente tutti
 i parametri di configurazione, in seguito la funzione
  \textit{execvp} che permetterà l'effetiva creazione del processo in questione che effettuerà le 
  seguenti operazioni: 
\begin{itemize}
    \item Registrare il segnale SIGINT che permetterà di gestire attraverso la combinazione ctrl+c da tastiera ,lo start and stop del processo a run-time.
    \item Preleva l'id della message queue su cui effettuera le limitazioni di attivazioni di processi atomo ,il numero di limitazioni è pari alla somma dei parametri di configurazione N\_NUOVI\_ATOMI + N\_ATOMI\_INIT. 
    \item Preleva l'id della message queue su cui invierà i dati necessari per il calcolo delle statistiche della simulazione. 
    \item Preleva una quantità di energià pari a ENERGY\_DEMAND.
\end{itemize}
\subsubsection{Meccanismo di assorbimento dell'energia e limitazione delle fissioni}
Il prelievo di energià avviene attraverso l'utilizzo di una shared memeory su cui viene scritto il quantitativo di energia che il processo inibitore assorbe ogni secondo. 
\newline
La limitazione delle fissioni di processi atomi avviene attraverso una meccanismo di scrittura sulla message queue che è in comunicazione tra processo attivatore e processo atomo , in modo tale da aumentare il numero di messaggi contenti il valore 0 ,che appunto non permette ad un processo atomo di effettuare l'operazione di fissione. 
\subsubsection{Possibilità di fermare e far riprendere il processo a discrezione dell'utente}
Il meccanismo progettato per permettere al processo inibitore di bloccare e riprendere la proprio esecuzione da input da tastiera a run-time consiste nel mascheramente del segnale SIGINT attraverso un nuovo gestore dei segnale che se riceeve proprio quest'ultimo controlla attraverso un flag se il processo è in esecuzione o meno ,in caso il processo inibitore fosse in esecuzione attraverso la combinazione ctrl+c da tastiera verrà inviato al processo inibitore un segnale SIGSTOP .
Nel caso invece che il processo inibitore è in fase di stop e viene ricevuta la combinazione ctrl+c da input da tastiera viene inviato attraverso funzione kill un segnale SIGCONT modificando il flag che identifica lo stato del processo. 